
\chapter{Harmonic Analysis of Functions}
\label{ch:harmonic}

In this chapter, we describe some tools for analyzing functions over probability
spaces, for which the sample space exhibits a field structure, and the 
distribution is defined in terms of the field operations.  In this case, we can 
obtain decompositions with special properties, that helps in 
analysing such distributions. We will be concerned with
the sample space which is the vector space of functions of the 
form $f:\F_p^R \rightarrow \F_p$ ($p$ is a prime).
 We  extend these decompositions and the associated results
 to the case when the domain is the subspace of low degree
 polynomials over $\F_p$. We prove an  analogue of  hypercontractivity
  for functions over this subspace (\lref[Lemma]{thm:hyp-for-poly}), by 
 reducing it to the result  on the full space (\lref[Lemma]{thm:mov-to-poly}).
 This enables us to prove an analogue of a
  result by Alon \etal ~\cite{AlonDFS2004} (\lref[Lemma]{thm:dict}), to 
  functions on the subspace (\lref[Lemma]{thm:derand-dict}). We use
these results later in \lref[Chapter]{ch:graph-prod}, for proving some
derandomized graph product results.
  
\section{Harmonic Analysis for Fields}
\label{sec:harmonic}

Consider the probability space $(\F_p,\mu)$ where $\mu$ is the uniform
probability measure over $\F_p$. We will be working with functions of the form
$A:\F_p^R \rightarrow \C$. Note that all such functions form a vector space over
$\C$ with dimension $p^R$. Characters are a natural orthogonal basis for this
vector space.

\begin{definition}[Character] A \emph{character} of $\F_p^R$ is a function
$\chi:\F_p^R \rightarrow \C$ such that 
$$\chi(0) = 1~~~~ \text{ and }~~~~ \forall f,g
\in \F_p^R,~ \chi(f+g) = \chi(f)\chi(g).$$ 
\end{definition}

The following lists the basic properties of characters, which can
be verified easily.

\begin{observation} \label{lem:fourier} 
Let $\{1,\omega,\cdots,\omega^{p-1}\}$ be the
$p$th roots of unity and for $\beta,f \in \F_p^R$, 
$$\chi_\beta(f) := \omega^{ \beta \cdot f } ~~~~\text{where}~~~~ \beta \cdot f := \sum_{i=1}^R \beta_i f_i \mod p.$$ 
\begin{itemize} 
\item The characters of $\F_p^R$ are $\{ \chi_\beta : \beta \in \F_p^R\}$. 
\item Characters forms an orthonormal basis for the vector space of 
functions from $\F_p^R$ to $\C$, under the inner product 
$$\langle A, B\rangle := \E_{f \in \F_p^R}\left[A(f)\overline{B(f)}\right].$$ 
\item Any function $A:\F_p^R \rightarrow \C$
can be uniquely decomposed as 
$$A(f) = \sum_{\beta \in \F_p^R}\widehat{A}(\beta) \chi_\beta(f)~~~~\text{
where} ~~~~\widehat{A}(\beta) := \E_{g \in \F_p^R} \left[A(g) 
\overline{\chi_\beta(g)}\right].$$ 
\item For any function $A:\F_p^R \rightarrow\C$, 
\begin{equation}
\label{eqn:parseval}
\sum_{\beta \in \F_p^R} |\widehat A(\beta)|^2 = \E_{f\in \F_p^R} \left[|A(f)|^2\right].
\end{equation}
\item For any function $A:\F_p^R\rightarrow \{1,\omega,\cdots,\omega^{p-1}\}$, 
\begin{equation}
\label{eqn:plancheral}
\sum_{\beta\in \F_p^R}|\widehat A(\beta)|^2 =1.
\end{equation}
\end{itemize} 
\end{observation}

\begin{remark}
\label{rem:char-f2}
Note that when $p=2$, the roots of unity are $\{-1,+1\}$. Then the characters
are real valued functions. It is easy to see that they form a orthonormal
basis for the real vector space of functions of the form $A:\F_2^R \rightarrow
\R$. Hence all the above properties hold with respect to this vector space
as well.
\end{remark}

\begin{definition}[Generalized Dictator] A \emph{generalized dictator} function 
$A:\F_p^R \rightarrow
\C$ is one that depends only on a single coordinate. That is, $\exists i \in
[R]$ such that for any $\beta \in \F_p^R$, if it has a non-zero entry in some
 coordinate $j\neq i$ then $\widehat A(\beta) = 0$. 
\end{definition}

\begin{definition}[Fourier degree] The \emph{Fourier degree} of a function
$A:\F_p^R \rightarrow \C$ is the smallest number $d$ such that $A$ can be
written as $$A = \sum_{\beta: |\beta| \leq d} \widehat A(\beta) \chi_\beta$$
where $|\beta|$ is number of coordinates $i$ where $\beta_i \neq 0$.
\end{definition}

An interesting fact about functions of the form $A:\F_2^R \rightarrow \{0,1\}$
is that, if $A$ has Fourier degree $1$ then $A$ is a generalized dictator function. 
The
following theorem over $\F_p$, says that if the sum of squares of absolute
values of Fourier coefficients of $A$ with $|\beta| >1$ is small, then it is
close to a generalized dictator function. It is a generalization of
the well known FKN Theorem (see \cite{FriedgutKN2002}), which
gives the result for $p=2$.

\begin{lemma}[Alon \etal\ \cite{AlonDFS2004}] \label{thm:dict} 
For every prime $p$, there is constant
$K$ (that depends on $p$) such that the following holds: 
If $A:\F_p^R \rightarrow \{0,1\}$ satisfies
$$\sum_{|\alpha| > 1} |\widehat A(\alpha)|^2 \leq \epsilon \text{ and } \widehat
A(0) =\delta$$ 
then there exists a generalized dictator $B:\F_p^R \rightarrow \{0,1\}$ such that 
$$\|A-B\|_2 \leq \frac{K\epsilon}{\delta -\delta^2 -\epsilon}.$$
\end{lemma} 
The above lemma is proved using the following hypercontractive
inequality. 
\begin{lemma}[Hypercontractivity] \label{thm:hyp} 
For every prime $p$, there is a
constant $C$ (that depends on $p$)
 such that for any function $A:\F_p^R \rightarrow \C$ with $\widehat
A(\alpha) = 0$ when $|\alpha| > t$, 
$$ \|A\|_4 \leq C^t \|A\|_2.$$ 
\end{lemma}

In the next section we will prove derandomized versions of the above lemmas.

\section{Polynomial Subspaces} 

Let $\Pe_{r,d}$ be the set of degree $d$
polynomials on $r$ variables over $\F_p$, with individual degrees $< p$
(the prime $p$ will be clear from the context). Let
$\mathfrak F_r := \Pe_{r,(p-1)r}$. Note that $\mathfrak F_r$ is the set of all
functions from $\F_p^r$ to $\F_p$. $\mathfrak F_r$ is a $\F_p$-vector space of
dimension $p^r$ and $\Pe_{r,d}$ is its subspace of dimension $r^{O(d)}$. The
\emph{Hamming distance} between $f$ and $g \in \mathfrak F_r$, denoted by
$\Delta(f,g)$, is the number of inputs on which $f$ and $g$ differ. When $S
\subseteq \mathfrak F_r$, $\Delta(f, S) := \min_{g\in S} \Delta(f,g)$. We say
$f$ is $\Delta$-far from $S$ if $\Delta(f,S) \geq \Delta$ and $f$ is
$\Delta$-close to $S$ otherwise. For a polynomial $\alpha \in \Pe_{r,d}$,
the support size of the polynomial is $|\alpha| := |\{ x: \alpha(x) \neq 0\}|$.  
Given $f,g, \in \mathfrak F_r$, the \emph{dot
product} between them is defined as 
$$ f \cdot g  := \sum_{x \in\F_p^r}f(x)g(x) \mod p.$$ 
For a subspace $S \subseteq \mathfrak F_r$, the \emph{dual subspace}
is defined as 
$$S^{\perp} := \{ g \in \mathfrak F_r : \forall f \in S,  g \cdot f  = 0 \}.$$ 
The following theorem relating dual spaces is well known.

\begin{lemma} \label{lem:dual}
$\Pe_{r,d}^\perp =\Pe_{r,(p-1)r-d-1}$
\end{lemma} 
\begin{proof}
First note that the dimensions of the two subspaces are equal by a counting argument.
Next we show that $\Pe_{r,d}^\perp \supseteq \Pe_{r,(p-1)r-d-1}$. We just need to show that for any
monomial of degree $(p-1)r - d -1$ with individual degrees $<p$, the dot product with any monomial of
degree $d$ with individual degrees $<p$ is $0$. The product of any such pair of monomials
is a monomial with total degree at most $(p-1)r - 1$, and hence has a variable with
degree $<p-1$. Without loss of generality, let this variable be $x_1$ with degree $t < p-1$. 
 Notice that $\sum_{x_1 \in \F_p} x_1^t = 0 \mod p$  and hence the dot product is $0$. 
\end{proof}

We need the following Schwartz-Zippel-like Lemma for degree $d$ polynomials. 

\begin{lemma}[Schwartz-Zippel lemma~{\cite[Lemma~3.2]{HaramatySS2013}}] 
\label{lem:SZ} Let $f\in \F_p[x_1,\cdots,x_r]$ be a
non-zero polynomial of degree at most $d$ with individual degrees at most $p-1$.
Then the support size ($|f| := |\{ x: f(x) \neq 0\}|$) satisfies
$$|f| \geq p^{r-d/(p-1)}.$$ 
\end{lemma}
 The following lemma is an easy consequence of \lref[Lemma]{lem:SZ}.
 
 \begin{lemma} \label{lem:low-deg-local-ind} Let $g$ be a
uniformly random polynomial from $\Pe_{r,d}$. Then its truth table as a random string of length $p^r$
over the alphabet $\F_p$, is $p^{\lfloor (d+1)/(p-1) \rfloor} - 1$-wise independent.
\end{lemma} 
\begin{proof} From  \lref[Lemma]{lem:SZ} and
\lref[Lemma]{lem:dual}, we know that any non-zero polynomial in $\Pe^{\perp}_{r,d} = \Pe_{r,(p-1)r - d -1}$
has support size at least $p^{\lfloor (d+1)/(p-1) \rfloor}$. Suppose there is a subset $S$
of size $p^{\lfloor (d+1)/(p-1) \rfloor}-1$, where $g$ is not uniform. Let $V \subseteq \F_p^{|S|}$
 be the set of restrictions of
truth tables of polynomials in $\Pe_{r,d}$ to $S$. Note that $V$ is a subspace. Since
the truth table of $g$ restricted to S is not uniformly distributed, the dimension 
of $V$ is $<|S|$. Then $V^{\perp} \subseteq \F_p^{|S|}$ is non-empty. Consider a
non-zero $v \in V^{\perp}$. Then the function $f$ which is zero outside $S$ and $f_{|S} = v$
corresponds to a non-zero polynomial which belongs to $\Pe^{\perp}_{r,d} = \Pe_{r,(p-1)r - d -1}$
with support $< p^{\lfloor (d+1)/(p-1) \rfloor}$ which is a contradiction.

\end{proof}


The following lemma is an easy consequence of \lref[Lemma]{lem:low-deg-local-ind} 
\begin{lemma}
\label{lem:interpol}
 Let $d>1$, $X$ be a set of $p^{d}-1$ points in $\F^r_p$ and
$f:X \rightarrow \F_p$ an arbitrary function. Then there exists a polynomial $q$
of degree at most $(p-1)d$ such that $q$ agrees with $f$ on all points in $X$.
\end{lemma} 
\begin{proof} 
By \lref[Lemma]{lem:low-deg-local-ind}, 
the truth table of a random polynomial $g$ of degree $(p-1)d$ is $p^d -1$-wise independent.
Hence $g_{|X} = f$ with non-zero probability.
 \end{proof} 

\section{Harmonic Analysis for Polynomial Subspaces} 
\label{sec:har-poly}

We  define a
orthonormal basis set of characters for the vector space of functions of the
form $A:\Pe_{r,d} \rightarrow \C$. 
\begin{definition}[Character] 
A \emph{character} of $\Pe_{r,d}$ is a function $\chi:\Pe_{r,d} \rightarrow \C$
such that 
$$\chi(0) = 1 \text{ and } \forall f,g \in \Pe_{r,d},~ \chi(f+g)=\chi(f)\chi(g).$$ 
\end{definition}

The following lists the basic properties of characters 
(similar to \lref[Observation]{lem:fourier}).  
 
\begin{observation}[{\cite[Section II C]{DinurG2013}}]
\label{lem:fourier-poly} 
Let $\{1,\omega,\cdots,\omega^{p-1}\}$ be the $p$th roots of
unity and for $\beta \in \mathfrak F_r, f \in \Pe_{r,d}$,

 $$\chi_\beta(f) := \omega^{ \beta \cdot f } ~~~~\text{where}
 ~~~~ \beta \cdot f := \sum_{x \in \F_p^r} \beta(x) f(x) \mod p.$$ 
 
\begin{itemize} \item The characters of
$\Pe_{r,d}$ are $\{ \chi_\beta : \beta \in \mathfrak F_r\}$. \item For any
$\beta,\beta' \in \mathfrak F_r$, $\chi_\beta = \chi_\beta'$ if and only if
$\beta-\beta' \in \Pe_{r,d}^\perp$. 
\item For $\beta \in \Pe_{r,d}^\perp$,
$\chi_\beta$ is the constant $1$ function. 
\item\label{item:minsup} $\forall
\beta, \exists \beta'$ such that $\beta-\beta' \in \Pe_{r,d}^\perp$ and
$|\beta'| = \Delta(\beta, \Pe_{r,d}^\perp)$ (i.e., the constant $0$
function is (one of) the closest function to $\beta'$ in $\Pe_{r,d}^\perp$). We
call such a $\beta'$ a minimum support function for the coset $\beta +
\Pe_{r,d}^\perp$.  
\item Characters forms an orthonormal basis for the vector
space of functions from $\Pe_{r,d}$ to $\C$, under the inner product 
$$\langle A,
B\rangle := \E_{f \in \Pe_{r,d}} \left[A(f)\overline{B(f)}\right].$$ 

\item Any function $A:\Pe_{r,d} \rightarrow \C$ can be uniquely decomposed as 
$$A(f) =\sum_{\beta \in \Lambda_{r,d}}\widehat{A}(\beta) \chi_\beta(f) 
~~~~ \text{where} ~~~~ \widehat{A}(\beta) := \E_{g \in \Pe_{r,d}} \left[A(g)
\overline{\chi_\beta(g)}\right]$$
 and $\Lambda_{r,d}$ is the set of minimum
support functions, one for each of the cosets in $\mathfrak
F_r/\Pe_{r,d}^\perp$, with ties broken arbitrarily. 
\item For any function $A:\Pe_{r,d} \rightarrow\C$, 
$$\sum_{\beta \in \Lambda_{r,d}} \left|\widehat A(\beta)\right|^2 = \E_{f\in \Pe_{r,d}}
\left[\left|A(f)\right|^2\right].$$ 
\item For any function $A:\Pe_{r,d}\rightarrow\{1,\omega,\cdots,\omega^{p-1}\}$, 
$$\sum_{\beta\in \Lambda_{r,d}}|\widehat A(\beta)|^2 =1.$$ 
\end{itemize} 
\end{observation} 
The following lemma relates
characters over different domains related by co-ordinate projections.
\begin{lemma} \label{lem:char-projection} 
Let $m \leq r$ and $\pi:\F_p^r
\rightarrow \F_p^m$ be a (co-ordinate) projection i.e., there exist indices $1
\leq i_1 < \cdots < i_m \leq r$ such that $\pi(x_1,\dots,x_r) = (x_{i_1}, \cdots
,x_{i_m})$. Then for $f \in \Pe_{m,d}, ~\beta \in \Pe_{r,d}$,
$$\chi_\beta(f\circ \pi)= \chi_{\pi_p(\beta)}(f),$$ 
where $\pi_p(\beta)(y):=\sum_{x \in \pi^{-1}(y)} \beta(x)$. 
\end{lemma}
\begin{proof}
Without loss of generality, let $\{i_1, \cdots, i_m\} = \{1,\cdots, m\}$. Then
\begin{align*}
 \chi_\beta(f\circ \pi) &= \omega^{\sum_{x \in \F_p^r} f\circ \pi(x) \cdot \beta(x)} \\
&= \omega^{\sum_{(x_1,\cdots,x_m) \in \F_p^m} f(x_1,\cdots,x_m)\cdot( \sum_{(x_{m+1},\cdots, x_n)} \beta(x))}\\
&= \chi_{\pi_p(\beta)}(f)
\end{align*}
\end{proof}

Influence and generalized dictators can be defined for functions on polynomial subspaces
similar to the product setting.

\begin{definition}[Influence] For a function $A:\Pe_{r,d} \rightarrow \C$ and a
number $k < p^{\lfloor (d+1)/(p-1) \rfloor}/2$, the degree $k$ influence of $a\in
\F_p^r$ is defined as 
$$\Inf^{\leq k}_a(A) = \sum_{\beta \in \Lambda_{r,d}:\beta(a) 
\neq 0 \text{ and } |\beta| \leq k} |\widehat A(\beta)|^2.$$ 
\end{definition}


\begin{definition}[Generalized Dictator] A function $A:\Pe_{r,d} \rightarrow \C$ is a
generalized dictator if there exists $x\in \F_p^r$ and $\widehat A_0, \widehat A_{1}, \cdots,
\widehat A_{p-1} \in \C$ such that $A$ can be written as $A = \widehat A_0 +
\sum_{i=1}^{p-1} \widehat A_{i}\chi_{ie_x}$ where $e_x:\F_p^r
\rightarrow \F_p$ the indicator function for $x$. 
\end{definition}

\begin{lemma}\label{lem:level1-dict}
Let $A:\Pe_{r,d} \rightarrow \{0,1\}$ be such that all non-zero
Fourier coefficients have support size $\leq 1$. Then $A$ is a generalized dicator.
\end{lemma}
\begin{proof}
The proof is similar to the proof of \cite[Lemma 2.3]{AlonDFS2004}. Consider the function $(A(f))^2$. Since $A$ is $\{0,1\}$ valued $(A(f))^2 = A(f)$. Equation the Fourier coefficients on both sides will give that there is an $x\in \F_p^r$ such that $A(f)$ only depends on $f(x)$.
\end{proof}




We prove an analogue of \lref[Theorem]{thm:hyp}, to functions over polynomial
subspaces. 

\begin{lemma}
\label{thm:hyp-for-poly} For every prime $p$, 
there is a constant $C$ such that for $4t\leq p^{d-1}$
and any function $A:\Pe_{r,(p-1)d} \rightarrow \C$ with $\widehat A_\alpha = 0$ when
$|\alpha| > t$, $$\|A\|_4 \leq C^t \|A\|_2.$$ 
\end{lemma} 
\begin{proof} Follows
from \lref[Lemma]{thm:mov-to-poly} and \lref[Lemma]{thm:hyp}. 
\end{proof}

We prove an analogue of \lref[Theorem]{thm:dict}, to functions over polynomial
subspaces. 
\begin{lemma} \label{thm:derand-dict} 
For every prime $p$, there is a constant $K$ such
that the following holds: If $A:\Pe_{r,(p-1)d} \rightarrow \{0,1\}$ satisfies
$$\sum_{|\alpha| > 1} |\widehat A(\alpha)|^2 \leq \epsilon \text{ and } \widehat
A(0) =\delta$$ 
then there exists a generalized dictator $B:\Pe_{r,(p-1)d} \rightarrow \{0,1\}$
such that 
$$\|A-B\|^2_2 \leq \frac{K\epsilon}{\delta -\delta^2-\epsilon}.$$
\end{lemma} 
\begin{proof}
The proof of the lemma is similar to the proof of \cite[Lemma 2.4]{AlonDFS2004}.
Let $K = 2 + 32C^8$ where $C$ is the constant
from \lref[Lemma]{thm:hyp-for-poly}.
First if $\epsilon \geq \frac{1}{32C^8}$, then the lemma is true. 
This is because, for any $B:\Pe_{r,(p-1)d}\rightarrow\{0,1\}$, 
$A-B$ is a $\{-1, 0,1\}$ valued function and $\| A-B\|^2_2 \leq 1$.

Now assume $\epsilon < \frac{1}{32C^8}$. Let
$$A_S = \sum_{|\alpha| \leq 1} \widehat A(\alpha) \chi_\alpha \mbox{ and } A_L = \sum_{|\alpha| > 1} \widehat A(\alpha) \chi_\alpha.$$
($S$, $L$ stands for small and large). If $A_S$ were Boolean, it  has to be a dictator by \lref[Lemma]{lem:level1-dict}. Then the lemma follows by taking $B=A_S$. Consider the following function
which measures the farness of $A_S$ from being Boolean (it is identically $0$, for Boolean functions)
$$H := A_S^2 - A_S.$$
Since $A_S$ does not have any Fourier coeffcients with support $>1$, $H$ will have only Fourier coefficients  with $|\alpha$ to be $0,1$ and $2$.
Let $e_x$ be the function with $e_x(x)=1$ and $0$ otherwise. Then for $a,b \in F_p, x,y \in \F_p^r$
$$ \widehat H(ae_x + be_y) = 2 \widehat A(ae_x) \widehat A(be_y).$$
The following claim says that the norm of $H$ is small.
\begin{claim}\label{claim:h}
$$\|H\|^2 \leq 32C^8 \epsilon.$$
\end{claim}
The claim is proved later. Let $a_x := \sum_{ i \in \F_p} | \widehat A(i e_x)|$. Note that
\begin{equation}
\sum_{x,y \in \F_p^r, x\neq y} a_x a_y \leq \frac{\|H\|^2}{4} \leq 8 C^8 .
\end{equation}
Also from assumptions in the claim, 
\begin{equation}
\sum_{x \in \F_p^r} a_x = \delta - \delta^2 - \epsilon.
\end{equation}
Let $y$ be such that $a_y$ is maximal. Then
$$(\sum_x a_x)^2 \leq \sum_x a_x^2 + 16C^8 \leq a_y \sum_x a_x + 16C^8.$$
This gives that $a_y \geq \delta -\delta^2 -\epsilon(1+ 16C^8/(\delta -\delta^2 - \epsilon))$.
If $B' := \widehat A(0) + \sum_{i \in \F_p} \widehat A(i e_y) \chi_{i e_y}$, then 
we have that $\|A - B'\|^2_2 \leq \epsilon(1+ 16C^8/(\delta -\delta^2 - \epsilon)).$
Now rounding $B$ to the closest $[0,1]$ valued function $B'$ pointwise, we get that
$$\|A-B\|_2^2 \leq 2\|A-B'\|_2^2 \leq  \frac{K\epsilon}{\delta -\delta^2-\epsilon}.$$


\begin{proof}[Proof of Claim \ref{claim:h} ]
First notice,
$$H = A_S^2 - A_S = (A-A_L)^2- (A-A_L) = A_L^2 +A_L(1-2A).$$
Let $k=2C^4$ and $Z = \{f : | A_L(f) \leq k \sqrt{\epsilon}\}$. Since $\|A_L\|_2^2 \leq \epsilon$,
by a Markov argument, $\Pr_f[Z] \geq 1- 1/k^2$.  Also for every $f \in Z$,
$|H(f)| \leq 2 |A_L(f)| \leq 2k \sqrt{\epsilon}$. Since $H$ has only Fourier coefficients
with support size $0,1,2$, we can use \lref[Lemma]{thm:hyp-for-poly} with $t=2$. The claim follows from
the following
\begin{align*}
\|H\|_2^2 &= \E_f |H(f)|^2 = \Pr[Z] \E_{f \in Z} |H(f)|^2   + (1-\Pr[Z]) \E_{f \notin Z} |H(f)|^2\\
&\leq 4k^2\epsilon + \frac{1}{k^2} \sqrt{\E_{f \notin Z} |H(f)|^4}\\
&\leq 4k^2\epsilon + \frac{1}{k} \sqrt{\E_{f} |H(f)|^4}\\
&\leq 4k^2\epsilon + \frac{1}{k} C^4 \|H\|_2^2 \leq 32C^8\epsilon
\end{align*}
\end{proof}

\end{proof}
\begin{definition}[Lift]\label{def:lift} For a function $B:\Pe_{r,d}
\rightarrow \C$ with the Fourier decomposition $B= \sum_{\alpha \in
\Lambda_{r,d}} \widehat{B}(\alpha) \chi_\alpha$, the lift of $B$ denoted by $B'$
is a function $B':\Ef_r \rightarrow \C$ with the Fourier decomposition $B'=
\sum_{\alpha \in \Lambda_{r,d}} \widehat{B}(\alpha) \chi_\alpha$. In the
decomposition of $B'$, $\chi_\alpha$'s are functions with domain $\Ef_r$.
\end{definition} 
\begin{lemma} \label{thm:mov-to-poly} 
If $2kt \leq p^{d-1}$ and
$B:\Pe_{r,(p-1)d} \rightarrow \C$ be a function such that $\widehat B(\alpha) = 0$
when $|\alpha| > t$ then 
$$\|B\|_{2k} = \|B'\|_{2k}.$$ 
\end{lemma} 
\begin{proof}
From the \lref[Lemma]{lem:SZ} and \lref[Lemma]{lem:dual}, we have that $\forall
\alpha \in \Pe^{\perp}_{r,(p-1)d}\setminus \{0\},~|\alpha| > p^{d-1}$. So if
$\exists \{\alpha_i,\beta_i \}_{i \in [k]}$ with $|\alpha_i|,|\beta_i| \leq t$,
then \begin{equation} \label{eqn:small-sup-is-zero} \sum_{i \in [k]} \alpha_i
-\beta_i \in \Pe^{\perp}_{r,(p-1)d} \Rightarrow \sum_{i \in [k]} \alpha_i -\beta_i =
0. \end{equation} This is because $\sum_{i \in [t]} \alpha_i -\beta_i$ has
support size at most $2kt < p^{d-1}$. We use this fact to prove the theorem as
follows: \begin{align*} \|B\|_{2k}^{2k} & = \E_{f \in \Pe_{r,(p-1)d}} |B(f)|^{2k} 
= \E_{f \in \Pe_{r,(p-1)d}} \prod_{i \in [k]}B(f)\overline {B(f)}\\
&=\sum_{\alpha_1,\beta_1,\cdots,\alpha_k,\beta_k \in \Lambda_{r,(p-1)d}}
\left(\prod_{i \in [k]}\widehat B_{\alpha_i} \overline{\widehat
B_{\beta_i}}\right) \E_{f \in \Pe_{r,(p-1)d}} \prod_{i \in [k]} \chi_{\alpha_i}(f)
\overline{\chi_{\beta_i}(f)}\\
&=\sum_{\substack{\alpha_1,\beta_1,\cdots,\alpha_k,\beta_k \in \Lambda_{r,(p-1)d}\\
\sum_i \alpha_i -\beta_i \in \Pe^{\perp}_{r,(p-1)d}}}~ \prod_{i \in [k]}\widehat
B_{\alpha_i} \overline{\widehat B_{\beta_i}} \\
&=\sum_{\substack{\alpha_1,\beta_1,\cdots,\alpha_k,\beta_k \in \Lambda_{r,(p-1)d} \\
\sum_i \alpha_i -\beta_i = 0}}~ \prod_{i \in [k]}\widehat B_{\alpha_i}
\overline{\widehat B_{\beta_i}}~~(\text{ from }
\eqref{eqn:small-sup-is-zero}~)\\ &=
\sum_{\alpha_1,\beta_1,\cdots,\alpha_k,\beta_k\in \Lambda_{r,(p-1)d}} \left(\prod_{i
\in [k]}\widehat B_{\alpha_i} \overline{\widehat B_{\beta_i}}\right) \E_{f \in
\Ef_r} \prod_{i \in [k]} \chi_{\alpha_i}(f) \overline{\chi_{\beta_i}(f)}\\ &=
\E_{f \in \Ef_r} \prod_{i \in[k]}B'(f)\overline {B'(f)} = \E_{f \in \Ef_r}
|B'(f)|^{2k}=\|B'\|_{2k}^{2k} \end{align*} \end{proof}


\subsection{Folding over Subspace} 

\begin{definition}[Folded function over a subspace] 
For any set $S$, a function $A: \Pe_{r,(p-1)d} \rightarrow S$ is said
to be folded over a subspace $J \subseteq \Pe_{r,(p-1)d}$ if $A$ is constant
over cosets of $J$ in $\Pe_{r,(p-1)d}$. 
\end{definition} 
\begin{fact}
\label{fact:ideallift} Given a function $A:\Pe_{r,(p-1)d}/J \rightarrow S$ there
is a unique function $A':\Pe_{r,(p-1)d} \rightarrow S$ that is folded over $J$
such that for $g \in \Pe_{r,(p-1)d}, A'(g) = A(g + J)$. 
\end{fact} 
Given
$q_1,\cdots, q_k \in \Pe_{r,3(p-1)}$, let 
$$J(q_1,\dots,q_k): = \left\{ \sum_i r_i q_i : r_i \in \Pe_{r,(p-1)(d-3)}\right\}.$$ 
The following lemma shows that
if a function is folded over $J=J(q_1,\dots,q_k)$, then it cannot have weight on
small support characters that are non-zero on $J$ (this is a generalization of
the corresponding lemma by Dinur \& Guruswami~\cite{DinurG2013} to arbitrary
fields). 
\begin{lemma} \label{lem:goodsupport} Let $\beta \in \mathfrak F_r$ is
such that $|\supp(\beta)| < p^{d-3}$, and there exists $x \in \supp(\beta)$ with
$q_i(x) \neq 0$ for some $i$. Then if $A:\Pe_{r,d}\rightarrow \C$ is folded over
$J=J(q_1,\dots,q_k)$, then $\widehat{A}(\beta) = 0$. 
\end{lemma} 
\begin{proof}
Construct a polynomial $t$ which is zero at all points in support of $\beta$
except at $x$. From \lref[Lemma]{lem:interpol}, its possible to construct such a
polynomial of degree at most $(p-1)(d -3)$. Then we have that $tq_i \in J$ and
$\langle\beta,tq_i\rangle \neq 0$. Now 
\begin{align*}
\E_h\left[A(h)\chi_\beta(h)\right] &=\frac{1}{p} \E_h[ A(h)\chi_\beta(h) +
A(h+tq_i)\chi_\beta(h+tq_i)+\cdots \\
&\qquad + A(h+(p-1)tq_i)\chi_\beta(h+(p-1)tq_i)]\\
 &=\frac{1}{p}\E_h[ A(h)\chi_\beta(h) + A(h)\chi_\beta(h+tq_i) +\cdots\\
 & \qquad + A(h)\chi_\beta(h+(p-1)tq_i)]\\ 
 &=\frac{1}{p}\E_h[ A(h)\chi_\beta(h)(1+\chi_\beta(tq_i)+\cdots+\chi_\beta((p-1)tq_i)) ]\\ 
 &=\frac{1}{p}\E_h[ A(h)\chi_\beta(h)(1+\omega^{t(\beta \cdot q_i)}+\cdots+\omega^{(p-1)t(\beta \cdot q_i)}) ] =0\qquad \qedhere 
\end{align*} 
The last step is due to the fact that  the sum $(1+\omega+\cdots+ \omega^{p-1}) =0$. Since $t(\beta\cdot q_i) \neq 0$,
the previous equation contains this sum.
\end{proof}
