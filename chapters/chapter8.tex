
\chapter{Almost Coloring of Graphs} \label{ch:graph-hard} 

In this chapter, we describe our hardness results on almost
graph coloring (joint work with Dinur, Harsha \& Srinivasan \cite{DinurHSV2014}).
Recall the discussion
on graph coloring algorithms and hardness results from
\lref[Section]{sec:approx}. Given a $3$-colorable graph, best known algorithms
give a $n^{0.19996}$-coloring \cite{KawarabayashiT2014} and it is known that finding a $4$-coloring is
\NPHard~ \cite{GuruswamiK2004}. Assuming \UGC, Dinur, Mossel \& Regev~\cite{DinurMR2009}, showed that,
given an almost $3$-colorable graph, it is hard to find a $C$-coloring for any
constant $C$. Their exact result is as follows:

\begin{theorem}[Dinur, Mossel \& Regev \cite{DinurMR2009}]\label{thm:DMR} 
There
is a reduction from \UG\ instances $G$ with $n$ vertices and label set $[R]$ to
graphs $\cal G$ of size $n3^{R}$ such that 
\begin{itemize} 
\item \YES: If $G$ is an
instance of \UG\ with $\OPT(G)\geq 1-\epsilon$ then there is a subgraph
of $\cal G$ with fractional size $1-\poly(\epsilon)$ that is $3$-colorable.
\item \NO: If $G$ is an instance of \UG\ with soundness $\OPT(G) \leq \delta$ then $\cal G$
does not have any independent sets of fractional size $O(1/\log (1/\delta))$.
\end{itemize} 
\end{theorem} 

For any constant $C$, taking $\delta$ to be a small
enough constant will ensure that the chromatic number is $\geq C$ in the NO
case. For getting hardness results with super-constant $C$, requires us to have
sub-constant $\delta$ which depends on $n$. In \UGC, the relation between the
the label size $R$ and soundness $\delta$ is not specified. But $R =
\Omega(1/\delta)$, since a random labeling to a \UG\ instance, satisfies
$O(1/R)$ fraction of the constraints. Assuming \UGC, with $R = \poly(1/\delta)$
and $\delta = 1/\poly(\log n)$, will ensure that (1) in the NO case, the
chromatic number is $\Omega(\log \log n)$ and (2) size of $\calG$ is $\poly(n)$.

Dinur \& Shinkar~\cite{DinurS2010} improved the analysis of the reduction, to
show that there are no independent sets of fractional size $O(1/\poly(\delta))$
in the NO case. Using the same assumption mentioned earlier, this implies
hardness for chromatic number $\Omega(\poly(\log n))$. However for these
results to hold, the alphabet size $R$ has to be $O(\log n)$.

In this chapter, we give a more efficient reduction, which ensures that the size
of $\calG$ remains small even when $R= 2^{2^{O(\sqrt{\log \log n})}}$. The reduction
of Dinur, Mossel \& Regev~\cite{DinurMR2009}, used the graph product described in
\lref[Section]{sec:graph-prod} as a gadget. This is the reason why the size of
$\calG$ is $n3^R$. We replace this gadget by the derandomized graph product
construction from \lref[Section]{sec:derand-graph-prod}. This ensures that the
size of $\calG$ is $n3^{\poly_\delta(\log R)}$. Hence the reduction remains
polynomial time, even when the alphabet size is much larger than $\log n$.

For getting the hardness result, we need to assume a conjecture similar to the
Unique Games Conjecture with specific parameters.

\begin{conjecture}[$(c(n),s(n),r(n))$-UG Conjecture]
 It is NP-Hard to
distinguish between unique label cover instances $(U,V,E,R,\Pi)$ on $n$ vertices
and $R=\F_3^{r(n)}$ from the following cases: 
\begin{itemize}
 \item YES Case :
There is a labeling and a set $S\subseteq V$ of size $(1-c(n))|V|$ such that all
edges between vertices in $S$ are satisfied. 
\item NO Case : For any labeling,
at most $s(n)$ fraction of edges are satisfied. 
\end{itemize} 
\end{conjecture}

Khot \& Regev~\cite{KhotR2008} proved that the Unique Games Conjecture implies
that for any constants $c, s \in (0,1/2)$ there is a constant $r$ such that
$(c,s,r)$-UG Conjecture is true. We also require that the constraints of the
Unique Games instance are full rank linear maps.

\begin{definition}[Linear constraint] 
A constraint $\pi:R \rightarrow L$ is a
linear constraint of iff $R = L=\F_3^r$, and $\pi$ is a linear map of rank $r$.
\end{definition}

\begin{theorem}\label{thm:col-hard} 
There is a reduction from $(c,s,r)$-Unique
Label Cover instances $G$ with $n$ vertices, label set $\F_3^r$ and linear
constraints to graphs $\cal G$ of size $n3^{r^{O(\log 1/\mu)}}$ where $\mu =
\poly(s)$ such that 
\begin{itemize} 
\item If $G$ belongs to the YES case of
$(c,s,r)$-UG then there is a subgraph of $\cal G$ with fractional size $1-c$
that is $3$-colorable. 
\item If $G$ belongs to the NO case of $(c,s,r)$-UG
Conjecture then $\cal G$ does not have any independent sets of fractional size
$\mu$. 
\end{itemize} 
\end{theorem} 

Due to the efficiency of reduction, we are
able to get hardness results even if the label cover instances had
super-polylogarithmic sized label sets of size at most $2^{2^{O(\sqrt{\log \log
n})}}$ while the reduction due to Dinur and Shinkar only worked if label set is
of size $O(\log^c n)$ for some constant $c$. However we get this improvement
only when soundness of the label cover $s(n) = 1/2^{O(\sqrt{\log \log n})}$.
 We remark that we can
improve the conclusion if \lref[Theorem]{thm:derand-ds} can be proved even when
$d = O(\log \log 1/\mu)$.


\begin{corollary}\label{cor:main} 
Let $c,s,r$ be functions such that
$s^{-1}(n),r(n) = 2^{O(\sqrt {\log \log n})}$. Assuming $(c,s,r)$-UG Conjecture
on instances with linear constraints, given a graph on $N$ vertices which has an
induced subgraph of relative size $1-c$ that is $3$-colorable, no polynomial
time algorithm can find an independent set of size $\poly(s(N))$.
\end{corollary}

\section{Reduction}

In this section we prove \lref[Theorem]{thm:col-hard}. Let $G =(U,V,R,E,\Pi)$ be
a unique games cover instance with label set $R=\F_3^r$ and the constraints
$\pi$ are full rank linear transformations. We will construct a graph
$\mathcal{G=(V,E)}$ with $\mathcal V= V\times \Pe_{r,2d}$, where $d$ is a
parameter to be fixed later. Let $T_{r,d}$ be the operator in
\lref[Definition]{def:ops}. There is an edge in $\mathcal G$ between $(v,f)$ and
$(w,g)$ if there is a $u\in U$ such that $(u,v), (u,w) \in E$ and
$T_{r,d}(f\circ\pi^{-1}_{u,v},g \circ \pi^{-1}_{u,w}) >0$, where $\pi_{u,v}$ is
the full rank linear map that maps a label of $v$ to label of $u$.

\begin{lemma}[Completeness]\label{lem:completeness}
 If $G$ belongs to the YES
case of $(c,s,r)$-UG Conjecture then $\cal G$ has a induced subgraph of relative
size $1-c$ that is $3$-colorable. 
\end{lemma} 
\begin{proof} 
Suppose the label
cover instance has a labeling $\ell: V \rightarrow \F_3^r$ and a set $S
\subseteq V, |S| = (1-c)|V|$, such that $\ell$ satisfies all the edges incident
on vertices in $S$. We will show that $A_v(f):=f(\ell(v))$ for $v\in V$, is a
$3$-coloring for the induced subgraph of $\cal G$ on the set $S \times
\Pe_{r,2d}$. For any $u \in U, v,w \in S$ having edges $(u,v),(u,w)\in E$,
consider the edge $((v,f),(w,g)) \in \mathcal E$. The colors given to the end
points are $f(\ell(v))$ and $g(\ell(w))$. Since
$T_{r,d}(f\circ\pi^{-1}_{u,v},g\circ \pi^{-1}_{u,w}) >0$, $$g\circ
\pi^{-1}_{u,w} = f\circ \pi^{-1}_{u,v} + a(p^2 +1) \text{ for some } p \in
\Pe^r_d, a \in \{1,2\}.$$ So $f(\ell(v))= f \circ \pi_{u,v}^{-1}(\ell(u)) \neq
g\circ \pi_{u,w}^{-1}(\ell(u)) = g (\ell(w))$.
\end{proof}

\begin{lemma}[Soundness]\label{lem:soundness} 
If $G$ belongs to the NO case of
$(c,s,r)$-UG Conjecture, $\cal G$ has an independent set of relative size $\mu$
and $d= O(\log 1/\mu)$ then $ \mu \leq \poly(s(n)).$ 
\end{lemma} 
\begin{proof}
Let $I_v:\Pe_{r,2d} \rightarrow \{0,1\}$ be the indicator function of $I$
restricted to vertices in $\mathcal V$ corresponding to $v\in V$. Let $J=\{v \in
V: \E_{f \in \Pe_{r,2d}} I_v(f) \geq \mu/2\}$. Then we have that $|J|/|V| \geq
\mu/2$. For $v\in J$, let $L(v) = \{x \in \F_3^{r}: \Inf_x^{\leq k}(I_v) >
\delta \}$ where $\delta = \poly(\mu), k = O(\log 1/\mu)$ are parameters from
\lref[Theorem]{thm:derand-ds}. Note that $|L(v)| \leq k/\delta$, since the sum
of all degree $k$ influences is at most $k$. 
\begin{claim}\label{claim:sound}
Let $v,w \in J$ and $(u,v),(u,w) \in E$. Then there exists $a \in L(v), b \in
L(w)$ such that $\pi_{u,v}(a)= \pi_{u,w}(b)$. 
\end{claim} 
\begin{proof}
 Let
$A(f):= I_v(f\circ\pi^{-1}_{u,v})$, $B(g):= I_w(g\circ \pi^{-1}_{u,w})$. Since
$I$ is an independent set, if $(v,f\circ\pi^{-1}_{u,v}),
(w,g\circ\pi^{-1}_{u,w}) \in I$, then
$T_{r,d}(f\circ\pi^{-1}_{u,v},g\circ\pi^{-1}_{u,w}) =0$, which gives that
\begin{equation} 
\langle A, T_{r,d} B\rangle = 0 
\end{equation} 
From
\lref[Theorem]{thm:derand-ds}, there is some $c \in \F_3^r$ such that
$\Inf^{\leq k}_c( A),\Inf^{\leq k}_c(B) > \delta$, which gives that
$\pi_{u,v}^{-1}(c) \in L(v)$ and $\pi_{u,w}^{-1}(c) \in L(w)$. 
\end{proof} 
Now
consider the randomized partial labeling $L'$ to $G$, where for $v \in J$,
$L'(v)$ is chosen randomly from $L(v)$ and for $u\in U$, choose a random
neighbor $w \in J$ (if it exists), a random label $a \in L(w)$ and set $L(u)=
\pi_{u,w}^{-1}(a)$. For any $v \in J$, any edge $(u,v)$, the probability of it
being satisfied by $L'$ is $\mu^2/k^2 = \poly(\mu)$, because of
\lref[Claim]{claim:sound}.

\end{proof}

\begin{proof}[Proof of Theorem \ref{thm:col-hard}] 
The size of $\mathcal G$
denoted by $N$ is at most $ n 3^{r^{O(d)}}$. Substituting $r= 2^{O(\sqrt{\log
\log n})}, d = \log 1/\mu \leq O(\sqrt{\log \log n})$, we get that $N =
\poly(n)$ and hence the reduction is polynomial time. 
\end{proof}
