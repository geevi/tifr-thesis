
\chapter{PCPs \& Hardness of Approximation}

In this chapter, we will review some of the fundamental results in hardness of
approximation. One of the main results is an alternate characterization for \NP.
Recall the definition of \NP\ (\lref[Definition]{def:np}).
For the verifier of an \NP\ problem, the proof length is at most a polynomial in
$n$. The alternate characterization deals with probabilistic verifiers, that
query only a few bits in the proof.

\begin{definition}[{$\PCP_{c,s}[t,q,R]$}] 
A $\PCP_{c,s}[t(n),q(n), R]$-Verifier (\PCP\ stands for Probabilistically 
Checkable Proofs)
for a decision problem $L$, is an algorithm $V$
which takes an input $x$, a random string
$y \in \{0,1\}^{t(n)}$, has oracle access to a proof $\pi$ over alphabet $[R]$ of length $2^{t(n)}$
 and satisfies the following properties:
 \begin{itemize}
 \item $V$ runs in polynomial time in the length of $x$ for all $y,\pi$.
\item $V$ queries $\pi$ in at most $q(n)$ bits on all inputs. 
\item Completeness: If $x \in L$ then there exists $\pi$ such that $\Pr_y[V(x,\pi)=1]\geq c$.
\item Soundness : If $x \notin L$ then for any $\pi$, $\Pr_y[V(x, \pi)=1] \leq
s$.
 \end{itemize} 
 $\PCP_{c,s}[t,q,R]$ is the class of decision problems that have
a $ \PCP_{c,s}[t,q,R]$-Verifier.
 \end{definition} 
 It is not difficult to see that,
for any $q$, constant $R$ and $s<c$, $$\PCP_{c,s}[O(\log n),q,R] \subseteq
\NP.$$ A major breakthrough in PCP characterizations, which lead to many
hardness of approximation results was the following theorem due to Arora \&
Safra~\cite{AroraS1998} and Arora \etal~\cite{AroraLMSS1998}. 
\begin{theorem}[PCP Theorem] 
There exists constant $s <1, q,R$ such that $$\NP \subseteq
\PCP_{1,s}[O(\log n),q,R].$$ 
\end{theorem} 
In the above form, it is not clear
how the theorem might be useful in hardness of approximation results. The
theorem can be stated equivalently as a hardness of approximation result.

\begin{theorem}[Hardness of Approximating \MAXTSAT] There exists a constant $s
<1$, such that it is \NPHard\ to distinguish satisfiable \MAXTSAT\ instances,
from ones for which any assignment can satisfy at most $s$ fraction of the
clauses. \end{theorem}

It is easy to see that the above theorem is equivalent to having a
$\PCP_{1,s}[O(\log n),3,2]$-Verifier for \MAXTSAT, whose checks are \MAXTSAT\
clauses. The equivalence follows from viewing such verifiers as 
\MAXTSAT\ instances and vice versa.
 Any such verifier could be converted to a \MAXTSAT\ instance.
The instance is obtained by adding a variable for every bit of the proof,
and adding a \MAXTSAT\ clause for every check the verifier makes.
Any \MAXTSAT\ instance, has a trivial \PCP\ verification procedure,
whose proof is the assignment and the verifier chooses a random clause,
and checks if it is satisfied by the assignment.


\section{Label Cover}

The PCP theorem stated in terms hardness of approximation of \MAXTSAT, does not
give optimal inapproximability results (see H\aa stad \cite{Hastad2001}). 
To get strong results, it is used in
conjunction with the parallel repetition theorem of Raz~\cite{Raz1998}. 
This strong version of PCP
theorem is usually stated in terms of the \LabelCover\ problem.

\begin{definition}[\LabelCover]
	
	\label{def:label-cover} An instance $G=(U,V,E,L,
	R,\{\pi_e\}_{e\in E})$ of the {\LC} constraint satisfaction
	problem consists of a bi-regular bipartite graph $(U,V,E)$,
        two sets of alphabets $R$ and $L$ and a projection map $\pi_e : R \rightarrow
        L$ for every edge $e\in E$.
        Given a labeling $\ell : U \rightarrow R, \ell:V \rightarrow
        L$, an edge $e = (u,v)$ is said to be satisfied by $\ell$ if
        $\pi_e(\ell(v)) = \ell(u)$. 

        $G$ is said to be \emph{at most $\delta$-satisfiable} if every
        labeling satisfies at most a $\delta$ fraction of the
        edges. 
        
	An instance of \UG\ is a label cover instance where $L=R$ and the constraints 
	$\pi$ are permutations.
\end{definition}


We consider label cover instances obtained from $\TSAT$ instances in the
following natural manner. 
\begin{definition}[$r$-repeated label cover]
\label{def:label-cover} 
Let $\phi$ be a $\TSAT$ instance with $X$ as the set of
variables and $C$ the set of clauses. The $r$-repeated bipartite label cover
instance $I(\phi)$ is specified by: 
\begin{itemize} 
\item A graph $G:=(U,V,E)$,
where $U:=C^r, V:=X^r$. \item $\Sigma_U := \{0,1\}^{3r},\Sigma_V := \{0,1\}^r$.
\item There is an edge $(u,v) \in E$ if the tuple of variables $v$ can be
obtained from the tuple of clauses $u$ by replacing each clause by a variable in
it. 
\item The constraint $\pi_{uv}:\{0,1\}^{3r}\rightarrow \{0,1\}^{r}$ is
simply the projection of the assignments on $3r$ variables in all the clauses in
$u$ to the assignments on the $r$ variables in $v$. \item For each $u$ there is
a set of $r$ functions $\{f^u_i:\{0,1\}^{3r} \rightarrow \{0,1\} \}_{i=1}^r$
such that $f^u_i(a)=0$ iff the assignment $a$ satisfies the $i$th clause in $u$.
Note that $f^u_i$ depends only on the $3$ variables in the $i$th clause.
\end{itemize} 
A labeling $L_U:U\rightarrow \Sigma_U,L_V:V\rightarrow \Sigma_V$
satisfies an edge $(u,v)$ iff $\pi_{uv}(L_U(u))=L_V(v)$ and $L_U(u)$ satisfies
all the clauses in $u$. Let $\OPT(I(\phi))$ be the maximal fraction of
constraints that can be satisfied by any labeling. 
\end{definition} 
The
following theorem is obtained by applying  parallel repetition
theorem of Raz~\cite{Raz1998} with $r$ repetitions on hard instances of
\MAXTSAT\ where each variable occurs the same number of
times (see Feige's result~\cite{Feige1998}) and a structural property proved by
H\aa stad \cite[Lemma 6.9]{Hastad2001}. 

\begin{theorem} \label{thm:label-cover} 
There is an
algorithm which on input a \TSAT\ instance $\phi$ and $r\in \N$ outputs an
$r$-repeated label cover instance $I(\phi)$ in time $n^{O(r)}$ with the
following properties. 
\begin{itemize} 
\item Completeness: If $\phi \in \TSAT$, then $\OPT(I(\phi))=1$. 
\item Soundnes: If $\phi \notin \TSAT$, then $\OPT(I(\phi)) \leq
2^{-\epsilon_0 r}$ for some universal constant $\epsilon_0\in (0,1)$.
\item Smooth Projections: 		
$$\forall v \in V, \alpha \subset R, \qquad Pr_u \left[ |\pi_{uv}(\alpha)| <|\alpha|^{c_0}\right] \leq \frac{1}{|\alpha|^{c_0}}.$$
\end{itemize} 
Moreover, the underlying graph $G$ is both left and right regular.
\end{theorem}


 For our hardness results for
$3$-uniform $3$-colorable hypergraphs, we need a multipartite version of label
cover, satisfying a smoothness condition. 
\begin{definition}[\cite{Khot2002b}]
Let $I$ be a bipartite label cover instance specified by
$\left((U,V,E),\Sigma_U,\Sigma_V,\Pi\right)$. Then $I$ is $\eta$-\emph{smooth}
iff for every $u \in U$ and two distinct labels $a,b \in \Sigma_U$ 
$$\Pr_v[\pi_{uv}(a) = \pi_{uv}(b) ] \leq \eta,$$ 
where $v$ is a random neighbour of $u$.
\end{definition} 
\begin{definition}[$r$-repeated $\ell$-layered $\eta$-smooth label cover]
\label{def:multilayer} 
Let $T:=\lceil\ell/\eta\rceil$ and $\phi$ be
a \TSAT\ instance with $X$ as the set of variables and $C$ the set of clauses.
The $r$-repeated $\ell$-layered $\eta$-smooth label cover instance $I(\phi)$ is
specified by: 
\begin{itemize} 
\item An $\ell$-partite graph with vertex sets
$V_0, \cdots V_{\ell-1}$. Elements of $V_i$ are tuples of the form $(C',X')$
where $C'$ is a set of $(T+\ell - i)r$ clauses and $X'$ is a a set of $ir$
variables. 
\item $\Sigma_{V_i} := \{0,1\}^{m_i}$ where $m_i:={3(T+\ell - i)r
+ir}$ which corresponds to all Boolean assignments to the clauses and variables
corresponding to a vertex in layer $V_i$. 
\item For $0 \leq i < j < \ell$,
$E_{ij} \subseteq V_i \times V_j$ denotes the set of edges between layers $V_i$
and $V_j$. For $v_i \in V_i, v_j \in V_j$, there is an edge $(v_i,v_j) \in
E_{ij}$ iff $v_j$ can be obtained from $v_i$ by replacing some $(j-i)r$ clauses
in $v_i$ with variables occurring in the clauses respectively. 
\item The
constraint $\pi_{v_i v_j}$ is the projection of assignments for clauses and
variables in $v_i$ to that of $v_j$. 
\item For each $i <\ell$, $v_i \in V_i$,
there are $(T+\ell - i)r$ functions $f_j^{v_i}:\{0,1\}^{3(T+\ell - i)r
+ir}\rightarrow \{0,1\}$, one for each clause $j$ in $v_i$ such that
$f_j^{v_i}(a)=0$ iff $a$ satisfies the clause $j$. This function only depends on
the $3$ coordinates in $j$. \end{itemize} Given a labeling $L_i:V_i\rightarrow
\Sigma_{V_i}$ for all the vertices, an edge $(v_i,v_j) \in E_{ij}$ is satisfied
iff $L_i(v_i)$ satisfies all the clauses in $v_i$, $L_j(v_j)$ satisfies all the
clauses in $v_j$ and $\pi_{v_i v_j}(L_i(v_i)) = L_j(v_j)$. Let
$\OPT_{ij}(I(\phi))$ be the maximum fraction of edges in $E_{ij}$ that can be
satisfied by any labeling. 
\end{definition} 
The following theorem was proved by
Dinur~\etal~\cite{DinurGKR2005} in the context of hypergraph vertex cover
inapproximability (also see results of Dinur, Regev \& Smyth~\cite{DinurRS2005}). 
\begin{theorem}
\label{thm:layered-label-cover} 
There is an algorithm which on input a \TSAT\
instance $\phi$ and $\ell,r\in \N, \eta \in [0,1)$ outputs a $r$-repeated
$\ell$-layered $\eta$-smooth label cover instance $I(\phi)$ in time
$n^{O((1+1/\eta)\ell r)}$ with the following properties. 
\begin{enumerate} 
\item $\forall~ 0\leq i < j < \ell$, the bipartite label cover instance on
$I_{ij}=\left((V_i,V_j,E_{ij}),\Sigma_{V_i},\Sigma_{V_j},\Pi_{ij}\right)$ is
$\eta$-smooth. 
\item For $1<m<\ell$, any $m$ layers $0\leq i_1< \cdots <i_m\leq
\ell-1$, any $S_{i_j} \subseteq V_{i_j}$ such that $|S_{i_j}| \geq
\frac{2}{m}|V_{i_j}|$, there exists distinct ${i_j}$ and ${i_{j'}}$ such that
the fraction of edges between $S_{i_j}$ and $S_{i_{j'}}$ relative to
$E_{i_ji_{j'}}$ is at least $1/m^2$. 
\item If $\phi \in \TSAT$, then there is a
labeling for $I(\phi)$ that satisfies all the constraints. 
\item If $\phi \notin \TSAT$, then 
$$\OPT_{i,j}(I(\phi)) \leq 2^{-\Omega(r)}, \quad \forall 0\leq i <j \leq \ell.$$ 
\end{enumerate} 
\end{theorem}

\section{Unique Games Conjecture}

Khot observed~\cite{Khot2002} that if the label sets in the \LabelCover\
instance are the same and the projections are permutations, then
the hardness reductions could be simplified.
He made the
conjecture that \LabelCover\ is hard to approximate to any constant factor,
restricted to such kind of instances.
\begin{definition}[Unique Games Conjecture]
For every $\delta$ there exists a large enough $R$ such that it
is hard to distinguish between \UniqueGame\ instances $G$ have label
size  $R$ from the following cases.
\begin{itemize}
\item \YES\ case : $\OPT(G)\geq 1-\delta$
\item \NO\ case :  $\OPT \leq \delta$
\end{itemize}
\end{definition}

Starting with the work of Khot ~\cite{Khot2002}, it was shown that  
UGC, explains the lack of efficient approximation algorithms for a 
variety of problems (eg. Vertex Cover, MAX-CUT).

