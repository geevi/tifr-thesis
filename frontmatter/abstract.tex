% the abstract

The graph coloring problem is a notoriously hard problem, for which we do not
have efficient algorithms. A \emph{coloring} of a graph is an assignment of colors
to its vertices such that the end points of every edge have different
colors. A $k$\emph{-coloring} is a coloring that uses at most $k$ distinct colors.
The \emph{graph coloring} problem is to find a coloring that uses the
minimum number of colors. Given a $3$-colorable graph, the best known efficient
algorithms output an $n^{0.199\cdots}$-coloring. It is known that efficient
algorithms cannot find a $4$-coloring, assuming \P$\neq$\NP\ (such results are commonly known as \emph{hardness}
results). Hence there is a large gap
($n^{0.199\cdots}$ vs $4$) between what current algorithms can achieve and the
hardness results known.

In this thesis, we narrow the aforesaid gap for some generalizations of graph coloring, 
by giving improved hardness results (for exponentially better parameters in some cases). 
Some of our main results are as follows:

\begin{enumerate}

\item For the case of almost $3$-colorable graphs, we show hardness of finding
a $2^{\poly(\log \log n)}$-coloring, assuming a variant of the Unique Games
Conjecture (\UGC).

\item For the case of $3$-colorable $3$-uniform hypergraphs, we show 
quasi-\NP-hardness of finding a $2^{O(\log \log n/ \log \log \log n)}$-coloring.

\item For the case of $4$-colorable $4$-uniform hypergraphs, we show 
quasi-\NP-hardness of finding a $2^{(\log n)^{1/21}}$-coloring.

\item For the problem of the approximating the covering number of CSPs with
non-odd predicates, we show hardness of approximation to any constant factor,
assuming a variant of \UGC.

\end{enumerate}
